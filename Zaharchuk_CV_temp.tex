%!TEX TS-program = xelatex
%!TEX encoding = UTF-8 Unicode
% Awesome CV LaTeX Template for CV/Resume
%
% This template has been downloaded from:
% https://github.com/posquit0/Awesome-CV
%
% Author:
% Claud D. Park <posquit0.bj@gmail.com>
% http://www.posquit0.com
%
%
% Adapted to be an Rmarkdown template by Mitchell O'Hara-Wild
% 23 November 2018
%
% Template license:
% CC BY-SA 4.0 (https://creativecommons.org/licenses/by-sa/4.0/)
%
%-------------------------------------------------------------------------------
% CONFIGURATIONS
%-------------------------------------------------------------------------------
% A4 paper size by default, use 'letterpaper' for US letter
\documentclass[11pt,a4paper,]{awesome-cv}

% Configure page margins with geometry
\usepackage{geometry}
\geometry{left=1.4cm, top=.8cm, right=1.4cm, bottom=1.8cm, footskip=.5cm}


% Specify the location of the included fonts
\fontdir[fonts/]

% Color for highlights
% Awesome Colors: awesome-emerald, awesome-skyblue, awesome-red, awesome-pink, awesome-orange
%                 awesome-nephritis, awesome-concrete, awesome-darknight

\colorlet{awesome}{awesome-red}

% Colors for text
% Uncomment if you would like to specify your own color
% \definecolor{darktext}{HTML}{414141}
% \definecolor{text}{HTML}{333333}
% \definecolor{graytext}{HTML}{5D5D5D}
% \definecolor{lighttext}{HTML}{999999}

% Set false if you don't want to highlight section with awesome color
\setbool{acvSectionColorHighlight}{true}

% If you would like to change the social information separator from a pipe (|) to something else
\renewcommand{\acvHeaderSocialSep}{\quad\textbar\quad}

\def\endfirstpage{\newpage}

%-------------------------------------------------------------------------------
%	PERSONAL INFORMATION
%	Comment any of the lines below if they are not required
%-------------------------------------------------------------------------------
% Available options: circle|rectangle,edge/noedge,left/right

\name{Holly A. Zaharchuk}{}

\position{PhD Candidate (she/her)}
\address{University Park, Pennsylvania, United States}

\email{\href{mailto:hzaharchuk@psu.edu}{\nolinkurl{hzaharchuk@psu.edu}}}
\homepage{www.hzaharchuk.com}
\orcid{0000-0002-7693-7807}
\github{hollzzar}

% \gitlab{gitlab-id}
% \stackoverflow{SO-id}{SO-name}
% \skype{skype-id}
% \reddit{reddit-id}


\usepackage{booktabs}

\providecommand{\tightlist}{%
	\setlength{\itemsep}{0pt}\setlength{\parskip}{0pt}}

%------------------------------------------------------------------------------



% Pandoc CSL macros
\newlength{\cslhangindent}
\setlength{\cslhangindent}{1.5em}
\newlength{\csllabelwidth}
\setlength{\csllabelwidth}{2em}
\newenvironment{CSLReferences}[3] % #1 hanging-ident, #2 entry spacing
 {% don't indent paragraphs
  \setlength{\parindent}{0pt}
  % turn on hanging indent if param 1 is 1
  \ifodd #1 \everypar{\setlength{\hangindent}{\cslhangindent}}\ignorespaces\fi
  % set entry spacing
  \ifnum #2 > 0
  \setlength{\parskip}{#2\baselineskip}
  \fi
 }%
 {}
\usepackage{calc}
\newcommand{\CSLBlock}[1]{#1\hfill\break}
\newcommand{\CSLLeftMargin}[1]{\parbox[t]{\csllabelwidth}{\honortitlestyle{#1}}}
\newcommand{\CSLRightInline}[1]{\parbox[t]{\linewidth - \csllabelwidth}{\honordatestyle{#1}}}
\newcommand{\CSLIndent}[1]{\hspace{\cslhangindent}#1}

\begin{document}

% Print the header with above personal informations
% Give optional argument to change alignment(C: center, L: left, R: right)
\makecvheader

% Print the footer with 3 arguments(<left>, <center>, <right>)
% Leave any of these blank if they are not needed
% 2019-02-14 Chris Umphlett - add flexibility to the document name in footer, rather than have it be static Curriculum Vitae


%-------------------------------------------------------------------------------
%	CV/RESUME CONTENT
%	Each section is imported separately, open each file in turn to modify content
%------------------------------------------------------------------------------



\hypertarget{research-interests}{%
\section{Research interests}\label{research-interests}}

My research interests lie at the intersection of cognitive psychology
and sociolinguistics. I use neuropsychological (EEG) and behavioral
methods to understand how linguistic, cognitive, and social factors
interact during language processing. My particular areas of study
include phonetic and syntactic variation, bilingual code-switching, and
network models of language.

\hypertarget{education}{%
\section{Education}\label{education}}

\begin{cventries}
    \cventry{Ph.D., Psychology and Language Science, Specialization in Cognitive and Affective Neuroscience}{The Pennsylvania State University}{University Park, Pennsylvania}{2018 - present}{\begin{cvitems}
\item Advisor: Dr. Janet G. van Hell
\item Dissertation: \textit{Cross-language transfer in voice onset time: A window into perceptual adaptation in brain and behavior}
\end{cvitems}}
    \cventry{M.S., Psychology}{The Pennsylvania State University}{University Park, Pennsylvania}{2018 - 2020}{\begin{cvitems}
\item Advisor: Dr. Janet G. van Hell
\item Master's thesis: \textit{We ``might could'' revisit syntactic processing: Studying dialectal variation with event-related potentials}
\end{cvitems}}
    \cventry{B.A., Psychology, Minor in Linguistics}{The University of Chicago}{Chicago, Illinois}{2011 - 2015}{\begin{cvitems}
\item Advisor: Dr. Daniel Casasanto
\item Honors thesis: \textit{Does linguistic experience change mental representations of pitch?}
\end{cvitems}}
\end{cventries}

\hypertarget{honors-and-awards}{%
\section{Honors and awards}\label{honors-and-awards}}

\begin{cventries}
    \cventry{NSF grant supporting doctoral research on human language}{Doctoral Dissertation Research Improvement Grant (Ling-DDRI)}{The Pennsylvania State University}{Feb 2023 - Dec 2023}{}\vspace{-4.0mm}
    \cventry{NSF-funded fellowship to conduct field research in Colombia}{Partnerships for International Research and Education Fellowship (PIRE)}{The Pennsylvania State University}{May 2022 - Jun 2022}{}\vspace{-4.0mm}
    \cventry{Travel award to discuss dissertation and career plans with Dr. Emily Myers (UConn)}{Adele Miccio Memorial Travel Award}{The Pennsylvania State University}{Dec 2021}{}\vspace{-4.0mm}
    \cventry{Five-year fellowship for research doctoral students}{University Graduate Fellowship}{The Pennsylvania State University}{Aug 2018 - present}{}\vspace{-4.0mm}
    \cventry{Four-year merit scholarship}{University Scholar Award}{The University of Chicago}{Sep 2011 - Jun 2015}{}\vspace{-4.0mm}
\end{cventries}

\hypertarget{publications}{%
\section{Publications}\label{publications}}

\hypertarget{bibliography}{}
\leavevmode\vadjust pre{\hypertarget{ref-2}{}}%
Zaharchuk, H. A., \& Karuza, E. A. (2021). Multilayer networks: An
untapped tool for understanding bilingual neurocognition. \emph{Brain
and Language}, \emph{220}, 104977.

\leavevmode\vadjust pre{\hypertarget{ref-3}{}}%
Zaharchuk, H. A., Shevlin, A., \& Van Hell, J. G. (2021). Are our brains
more prescriptive than our mouths? Experience with dialectal variation
in syntax differentially impacts ERPs and behavior. \emph{Brain and
Language}, \emph{218}, 104949.

\leavevmode\vadjust pre{\hypertarget{ref-1}{}}%
Lucero, C., Zaharchuk, H., \& Casasanto, D. (2014). Beat gestures
facilitate speech production. In P. Bello, M. Guarini, M. McShane, \& B.
Scassellati (Eds.), \emph{Proceedings of the 36th Annual Meeting of the
Cognitive Science Society}. Red Hook, NY: Curran Associates.

\hypertarget{presentations}{%
\section{Presentations}\label{presentations}}

\hypertarget{bibliography}{}
\leavevmode\vadjust pre{\hypertarget{ref-9}{}}%
Zaharchuk, H. A., \& Van Hell, J. G. (2023, October). Investigating the
roles of acoustic-phonetic and lexico-semantic processing in resolving
perceptual ambiguity in spanish-accented english {[}Poster session{]}.
15th Annual Meeting of the Society for the Neurobiology of Language
(SNL), Marseille, France.

\leavevmode\vadjust pre{\hypertarget{ref-8}{}}%
Zaharchuk, H. A., Lipski, J., \& Van Hell, J. G. (2023, August). Mi casa
es tu posá: Exploring the bilingual mental lexicon in speakers of
spanish and palenquero {[}Poster session{]}. Architectures and
Mechanisms for Language Processing Conference (AMLaP 29), Donostia-San
Sebastián, Spain.

\leavevmode\vadjust pre{\hypertarget{ref-6}{}}%
Zaharchuk, H. A., Medina, V., Paterno, S., Viswanathan, N., \& Van Hell,
J. G. (2023, March). Sentential context impacts bilingual
speech-in-speech recognition {[}Poster session{]}. 36th Annual
Conference on Human Sentence Processing (HSP), Pittsburgh, Pennsylvania.

\leavevmode\vadjust pre{\hypertarget{ref-7}{}}%
Zaharchuk, H. A., Walker, A., \& Van Hell, J. G. (2023, March). Tracking
the time-course of cross-dialect comprehension with ERPs {[}Oral and
poster sessions{]}. 30th Annual Meeting of the Cognitive Neuroscience
Society (CNS), San Francisco, California.

\leavevmode\vadjust pre{\hypertarget{ref-5}{}}%
Zaharchuk, H. A., Shevlin, A., \& Van Hell, J. G. (2022, November).
Experiencia con variación dialéctica en sintaxis afecta el procesamiento
de lenguaje {[}Plenary session{]}. III Congreso Paraguayo de Lingüı́stica
Aplicada (CONPLA), Asunción, Paraguay.

\leavevmode\vadjust pre{\hypertarget{ref-4}{}}%
Walker, A., Zaharchuk, H. A., Fernandez, C. B., \& Van Hell, J. G.
(2021, August). The role of dialect expectation on lexical processing:
EEG evidence {[}Oral session{]}. 5th Variation and Language Processing
Conference (VALP5), Copenhagen, Denmark.

\leavevmode\vadjust pre{\hypertarget{ref-3}{}}%
Zaharchuk, H. A., Shevlin, A., \& Van Hell, J. G. (2020, May). We
{``might could''} revisit syntactic processing: Studying dialectal
variation with event-related potentials {[}Poster session{]}. 27th
Annual Meeting of the Cognitive Neuroscience Society (CNS), Boston,
Massachusetts.

\leavevmode\vadjust pre{\hypertarget{ref-2}{}}%
Zaharchuk, H. A., Shevlin, A., \& Van Hell, J. G. (2019, November).
Auditory comprehension of double versus single modal constructions in
Mainstream American English listeners {[}Poster session{]}. 60th Annual
Meeting of the Psychonomic Society, Montréal, Québec, Canada.

\leavevmode\vadjust pre{\hypertarget{ref-1}{}}%
Zaharchuk, H. A. (2015, June). Does linguistic experience change mental
representations of pitch? {[}Poster session{]}. University of Chicago
Undergraduate Research Symposium, Chicago, Illinois.

\hypertarget{invited-talks}{%
\section{Invited talks}\label{invited-talks}}

\begin{itemize}
\tightlist
\item
  Processing Variation (ProVAR) group led by Dr.~Marije Soto, Federal
  University of Rio de Janeiro
\item
  Psycholinguistic fieldwork guest lecture for Dr.~Colleen Balukas,
  University at Buffalo (SUNY)
\end{itemize}

\hypertarget{research-experience}{%
\section{Research experience}\label{research-experience}}

\begin{cventries}
    \cventry{Graduate student}{Bilingualism and Language Development Lab}{University Park, Pennsylvania}{Aug 2018 - present}{\begin{cvitems}
\item Principal investigator: Dr. Janet G. van Hell
\item Research areas: language processing and dialectal variation
\item Responsibilities: design, implement, and run EEG and behavioral experiments with E-Prime; process and analyze experimental data with R and MATLAB; mentor undergraduate research assistants; maintain lab equipment and supplies
\end{cvitems}}
    \cventry{Undergraduate research assistant}{Experience and Cognition Lab}{Chicago, Illinois}{Sep 2013 - Jun 2015}{\begin{cvitems}
\item Principal investigator: Dr. Daniel Casasanto
\item Research areas: embodied cognition and linguistic relativity
\item Responsibilities: train, run, and debrief participants; code and analyze experimental data with ELAN and R; create new experiments with Python
\end{cvitems}}
\end{cventries}

\hypertarget{professional-experience}{%
\section{Professional experience}\label{professional-experience}}

\begin{cventries}
    \cventry{Financial analyst}{Consumer Financial Protection Bureau}{Washington, District of Columbia}{Jul 2017 - Jul 2018}{\begin{cvitems}
\item Collaborated with the Consumer Education and Engagement Division to write a \href{https://www.consumerfinance.gov/about-us/blog/five-ways-banks-and-lenders-work-people-who-speak-or-understand-limited-english/}{\underline{spotlight report on language access}} for limited English proficient consumers as a member of the Language Access Task Force
\item Researched debt collection and debt relief, particularly in the credit card and student loan markets, for the Consumer Lending, Reporting, and Collections Markets team
\end{cvitems}}
    \cventry{Director's financial analyst}{Consumer Financial Protection Bureau}{Washington, District of Columbia}{Jul 2015 - Jul 2017}{\begin{cvitems}
\item Supported the policy and research teams in the Office of Fair Lending and Equal Opportunity
\item Performed business analytics, tracking, and reporting in the Supervision, Enforcement, and Fair Lending Front Office
\end{cvitems}}
    \cventry{Program and operations assistant}{Research Development Support (formerly Arete)}{Chicago, Illinois}{Feb 2013 - Jun 2015}{\begin{cvitems}
\item Partnered with The University of Chicago research administration support faculty members pursuing complex research initiatives
\item Provided program analysis, strategic planning, business metrics, and event support
\end{cvitems}}
\end{cventries}

\hypertarget{teaching-experience}{%
\section{Teaching experience}\label{teaching-experience}}

\begin{cventries}
    \cventry{Teaching assistant}{Psychology of language}{The Pennsylvania State University}{Jan 2021 - May 2021}{\begin{cvitems}
\item Instructor: Dr. Katharine Donnelly Adams
\item Seminar course covering auditory speech perception, visual word recognition, language representation, sentence comprehension and production, and language acquisition across the lifespan
\item Presented guest lecture on linguistic variation and the relation between production and comprehension
\end{cvitems}}
    \cventry{Teaching assistant}{Basic research methods in psychology}{The Pennsylvania State University}{Aug 2020 - Dec 2020}{\begin{cvitems}
\item Instructor: Dr. Nicholas Pearson
\item Introduction to experimental design, analysis, and presentation
\item Led weekly laboratory section to reinforce lecture concepts
\end{cvitems}}
    \cventry{Teaching assistant}{What makes us human?}{The Pennsylvania State University}{Jan 2020 - May 2020}{\begin{cvitems}
\item Instructor: Dr. Daniel Weiss
\item Senior seminar in comparative psychology
\end{cvitems}}
    \cventry{Teaching assistant}{Neurological bases of human behavior}{The Pennsylvania State University}{Aug 2019 - Dec 2019}{\begin{cvitems}
\item Instructor: Dr. Stephen Wilson
\item Introduction to neuroscience theories and methods
\item Presented guest lecture on auditory system and neuroanatomy of language production and comprehension
\end{cvitems}}
\end{cventries}

\hypertarget{outreach}{%
\section{Outreach}\label{outreach}}

\begin{cventries}
    \cventry{The Brain Bus coordinator}{Bilingualism and Language Development Lab}{University Park, Pennsylvania}{Aug 2018 - present}{\begin{cvitems}
\item Maintain \textit{The Brain Bus}, BiLD's mobile neuroscience lab for field research and language science public engagement
\item Participate in community outreach events, including as a volunteer judge at the Young Scholars of Western PA science fair
\end{cvitems}}
    \cventry{Graduate assistant}{Center for Language Science}{University Park, Pennsylvania}{Jan 2021 - May 2021}{\begin{cvitems}
\item Created public repository for research on linguistic discrimination in collaboration with fellow graduate assistants
\item Participated in ENVISION STEM career day for young women
\end{cvitems}}
    \cventry{Graduate assistant}{Center for Language Science}{University Park, Pennsylvania}{Aug 2019 - Dec 2019}{\begin{cvitems}
\item Managed the Center's social media presence to engage the wider regional community in language science research
\item Created and solicited content for the \href{https://sites.psu.edu/bilingualismmatters/winter-spring-2020/}{\underline{Winter/Spring 2020 Bilingualism Matters newsletter}}
\end{cvitems}}
\end{cventries}

\hypertarget{community-service}{%
\section{Community service}\label{community-service}}

\begin{cventries}
    \cventry{Mathematics tutor}{Next Step Public Charter School}{Washington, District of Columbia}{Jan 2017 - Jul 2018}{\begin{cvitems}
\item Volunteered as a mathematics tutor with the bilingual GED and ESL program for immigrant and other at-risk youth
\item Provided one-on-one assistance with practice problems and assignments once per week for 1-2 hours
\end{cvitems}}
    \cventry{Volunteer tax preparer}{Community Tax Aid}{Washington, District of Columbia}{Jan 2016 - Apr 2018}{\begin{cvitems}
\item Volunteered as an IRS-certified advanced volunteer tax preparer with the Volunteer Income Tax Assistance (VITA) program, which offers free tax help to individuals who make \$54,000 or less, individuals with disabilities, and limited English speaking individuals who need assistance in preparing their own tax returns
\item Provided tax preparation services once per week for 3-4 hours during tax season
\end{cvitems}}
\end{cventries}

\hypertarget{academic-service}{%
\section{Academic service}\label{academic-service}}

\begin{itemize}
\tightlist
\item
  Ad hoc reviewing

  \begin{itemize}
  \tightlist
  \item
    \emph{Brain and Language}
  \item
    \emph{Journal of Experimental Psychology: Learning, Memory, and
    Cognition}
  \item
    \emph{Scientific Reports}
  \end{itemize}
\item
  Workshops and tutorials

  \begin{itemize}
  \tightlist
  \item
    Leader: \href{https://github.com/hollzzar/markdown-tutorial}{R
    Markdown for Graduate Students} (2020),
    \href{https://github.com/hollzzar/reports-with-markdown}{Creating
    HTML Reports with R Markdown} (2020)
  \item
    Assistant:
    \href{https://github.com/psu-psychology/r-bootcamp-2019}{R Bootcamp}
    (2019) led by Dr.~Rick Gilmore
  \end{itemize}
\item
  \href{https://www.hzaharchuk.com/rmarkdown-guide/}{R Markdown for
  Psychology Graduate Students} resource
\end{itemize}

\hypertarget{skills}{%
\section{Skills}\label{skills}}

\begin{itemize}
\tightlist
\item
  Natural languages: English, Spanish (B2/Intermediate)
\item
  Programming languages: R, MATLAB, SQL, Python
\item
  Software: E-Prime, Praat, Microsoft Suite, Adobe Suite
\item
  Typesetting: R Markdown, \LaTeX
\end{itemize}

\hypertarget{affiliations}{%
\section{Affiliations}\label{affiliations}}

\begin{itemize}
\tightlist
\item
  Center for Language Science
\item
  Cognitive Neuroscience Society
\item
  Psychonomic Society
\item
  Society for the Neurobiology of Language
\item
  Women in Cognitive Science
\end{itemize}



\end{document}

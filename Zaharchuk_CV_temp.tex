%!TEX TS-program = xelatex
%!TEX encoding = UTF-8 Unicode
% Awesome CV LaTeX Template for CV/Resume
%
% This template has been downloaded from:
% https://github.com/posquit0/Awesome-CV
%
% Author:
% Claud D. Park <posquit0.bj@gmail.com>
% http://www.posquit0.com
%
%
% Adapted to be an Rmarkdown template by Mitchell O'Hara-Wild
% 23 November 2018
%
% Template license:
% CC BY-SA 4.0 (https://creativecommons.org/licenses/by-sa/4.0/)
%
%-------------------------------------------------------------------------------
% CONFIGURATIONS
%-------------------------------------------------------------------------------
% A4 paper size by default, use 'letterpaper' for US letter
\documentclass[11pt,a4paper,]{awesome-cv}

% Configure page margins with geometry
\usepackage{geometry}
\geometry{left=1.4cm, top=.8cm, right=1.4cm, bottom=1.8cm, footskip=.5cm}


% Specify the location of the included fonts
\fontdir[fonts/]

% Color for highlights
% Awesome Colors: awesome-emerald, awesome-skyblue, awesome-red, awesome-pink, awesome-orange
%                 awesome-nephritis, awesome-concrete, awesome-darknight

\colorlet{awesome}{awesome-red}

% Colors for text
% Uncomment if you would like to specify your own color
% \definecolor{darktext}{HTML}{414141}
% \definecolor{text}{HTML}{333333}
% \definecolor{graytext}{HTML}{5D5D5D}
% \definecolor{lighttext}{HTML}{999999}

% Set false if you don't want to highlight section with awesome color
\setbool{acvSectionColorHighlight}{true}

% If you would like to change the social information separator from a pipe (|) to something else
\renewcommand{\acvHeaderSocialSep}{\quad\textbar\quad}

\def\endfirstpage{\newpage}

%-------------------------------------------------------------------------------
%	PERSONAL INFORMATION
%	Comment any of the lines below if they are not required
%-------------------------------------------------------------------------------
% Available options: circle|rectangle,edge/noedge,left/right

\name{Holly A. Zaharchuk}{}

\position{Postdoctoral researcher (she/her)}
\address{Storrs, Connecticut, United States}

\email{\href{mailto:holly.zaharchuk@uconn.edu}{\nolinkurl{holly.zaharchuk@uconn.edu}}}
\homepage{www.hzaharchuk.com}
\orcid{0000-0002-7693-7807}
\github{hollzzar}

% \gitlab{gitlab-id}
% \stackoverflow{SO-id}{SO-name}
% \skype{skype-id}
% \reddit{reddit-id}


\usepackage{booktabs}

\providecommand{\tightlist}{%
	\setlength{\itemsep}{0pt}\setlength{\parskip}{0pt}}

%------------------------------------------------------------------------------


\usepackage{soul}

% Pandoc CSL macros
% definitions for citeproc citations
\NewDocumentCommand\citeproctext{}{}
\NewDocumentCommand\citeproc{mm}{%
  \begingroup\def\citeproctext{#2}\cite{#1}\endgroup}
\makeatletter
 % allow citations to break across lines
 \let\@cite@ofmt\@firstofone
 % avoid brackets around text for \cite:
 \def\@biblabel#1{}
 \def\@cite#1#2{{#1\if@tempswa , #2\fi}}
\makeatother
\newlength{\cslhangindent}
\setlength{\cslhangindent}{1.5em}
\newlength{\csllabelwidth}
\setlength{\csllabelwidth}{3em}
\newenvironment{CSLReferences}[2] % #1 hanging-indent, #2 entry-spacing
 {\begin{list}{}{%
  \setlength{\itemindent}{0pt}
  \setlength{\leftmargin}{0pt}
  \setlength{\parsep}{0pt}
  % turn on hanging indent if param 1 is 1
  \ifodd #1
   \setlength{\leftmargin}{\cslhangindent}
   \setlength{\itemindent}{-1\cslhangindent}
  \fi
  % set entry spacing
  \setlength{\itemsep}{#2\baselineskip}}}
 {\end{list}}

\usepackage{calc}
\newcommand{\CSLBlock}[1]{\hfill\break\parbox[t]{\linewidth}{\strut\ignorespaces#1\strut}}
\newcommand{\CSLLeftMargin}[1]{\parbox[t]{\csllabelwidth}{\strut#1\strut}}
\newcommand{\CSLRightInline}[1]{\parbox[t]{\linewidth - \csllabelwidth}{\strut#1\strut}}
\newcommand{\CSLIndent}[1]{\hspace{\cslhangindent}#1}

\begin{document}

% Print the header with above personal informations
% Give optional argument to change alignment(C: center, L: left, R: right)
\makecvheader

% Print the footer with 3 arguments(<left>, <center>, <right>)
% Leave any of these blank if they are not needed
% 2019-02-14 Chris Umphlett - add flexibility to the document name in footer, rather than have it be static Curriculum Vitae
\makecvfooter
  {January 2026}
    {Holly A. Zaharchuk~~~·~~~Curriculum vitae}
  {\thepage~ of \pageref{LastPage}~}


%-------------------------------------------------------------------------------
%	CV/RESUME CONTENT
%	Each section is imported separately, open each file in turn to modify content
%------------------------------------------------------------------------------



\section{Research interests}\label{research-interests}

How does the brain respond to linguistic variation? I use a combination
of neural (EEG, fMRI, TMS) and behavioral (eye-tracking, response
time/accuracy) methods to study the perception, representation, and
processing of diverse speech signals. I am particularly interested in
the effects of multilingual and multilectal experience on the language
system.

\section{Education}\label{education}

\begin{cventries}
    \cventry{Ph.D., Psychology and Language Science, Specialization in Cognitive and Affective Neuroscience}{The Pennsylvania State University}{University Park, Pennsylvania}{2018 - 2024}{\begin{cvitems}
\item Advisor: Dr. Janet G. van Hell
\item Dissertation: \textit{Cross-language transfer in voice onset time: A window into perceptual adaptation in brain and behavior}
\end{cvitems}}
    \cventry{M.S., Psychology}{The Pennsylvania State University}{University Park, Pennsylvania}{2018 - 2020}{\begin{cvitems}
\item Advisor: Dr. Janet G. van Hell
\item Master's thesis: \textit{We ``might could'' revisit syntactic processing: Studying dialectal variation with event-related potentials}
\end{cvitems}}
    \cventry{B.A., Psychology, Minor in Linguistics}{The University of Chicago}{Chicago, Illinois}{2011 - 2015}{\begin{cvitems}
\item Advisor: Dr. Daniel Casasanto
\item Honors thesis: \textit{Does linguistic experience change mental representations of pitch?}
\end{cvitems}}
\end{cventries}

\section{Academic employment}\label{academic-employment}

\begin{cventries}
    \cventry{Postdoctoral researcher}{Language and Brain Lab}{Storrs, Connecticut}{Aug 2024 - present}{\begin{cvitems}
\item Principal investigator: Dr. Emily B. Myers
\item Research areas: neurobiology of speech perception in young adults, older adults, and persons with aphasia
\item Responsibilities: design, implement, and analyze research projects investigating laterality, gradiency, and talker specificity in speech perception with fMRI, TMS, and eye-tracking (NIH R0I: 5R01DC013064-09)
\end{cvitems}}
    \cventry{Undergraduate research assistant}{Experience and Cognition Lab}{Chicago, Illinois}{Sep 2013 - Jun 2015}{\begin{cvitems}
\item Principal investigator: Dr. Daniel Casasanto
\item Research areas: embodied cognition and linguistic relativity
\item Responsibilities: collect, code, and analyze behavioral data
\end{cvitems}}
\end{cventries}

\section{Honors and awards}\label{honors-and-awards}

\begin{cventries}
    \cventry{NSF grant for doctoral research on human language (SBE-2234907)}{Doctoral Dissertation Research Improvement Grant (Ling-DDRI)}{The Pennsylvania State University}{Feb 2023 - Feb 2025}{}\vspace{-4.0mm}
    \cventry{NSF grant for international language research (OISE-1545900)}{Partnerships for International Research and Education Fellowship (PIRE)}{The Pennsylvania State University}{May 2022 - Jun 2022}{}\vspace{-4.0mm}
    \cventry{Travel award for professional development}{Adele Miccio Memorial Travel Award}{The Pennsylvania State University}{Dec 2021}{}\vspace{-4.0mm}
    \cventry{Competitive five-year fellowship for research doctoral students (100 awarded per year University-wide)}{University Graduate Fellowship}{The Pennsylvania State University}{Aug 2018 - Aug 2024}{}\vspace{-4.0mm}
    \cventry{Four-year merit scholarship for undergraduate students}{University Scholar Award}{The University of Chicago}{Sep 2011 - Jun 2015}{}\vspace{-4.0mm}
\end{cventries}

\newpage

\section{Peer-reviewed publications}\label{peer-reviewed-publications}

\phantomsection\label{refs-9c4c06008bf9fbee601256f43cfc1735}
\begin{CSLReferences}{1}{0}
\bibitem[\citeproctext]{ref-8}
Zaharchuk, H. A., Walker, A. J., Miller, A. R., Fernandez, C. B., \& Van
Hell, J. G. (2050). Expecting a challenge: A behavioral and
neurophysiological investigation of talker identity in cross-dialectal
speech perception. \emph{Journal of Laboratory Phonology}.

\bibitem[\citeproctext]{ref-4}
Miller, A. R., Jończyk, R., Zaharchuk, H. A., \& Van Hell, J. G. (2025).
Unlocking second language novel metaphor processing: Behavioral and ERP
insights from first and second‐language English users.
\emph{Psychophysiology}, \emph{62}(5), e70066.

\bibitem[\citeproctext]{ref-2}
Zaharchuk, H. A., \& Karuza, E. A. (2021). Multilayer networks: An
untapped tool for understanding bilingual neurocognition. \emph{Brain
and Language}, \emph{220}, 104977.

\bibitem[\citeproctext]{ref-3}
Zaharchuk, H. A., Shevlin, A., \& Van Hell, J. G. (2021). Are our brains
more prescriptive than our mouths? Experience with dialectal variation
in syntax differentially impacts ERPs and behavior. \emph{Brain and
Language}, \emph{218}, 104949.

\bibitem[\citeproctext]{ref-1}
Lucero, C., Zaharchuk, H., \& Casasanto, D. (2014). Beat gestures
facilitate speech production. In P. Bello, M. Guarini, M. McShane, \& B.
Scassellati (Eds.), \emph{Proceedings of the 36th Annual Meeting of the
Cognitive Science Society}. Red Hook, NY: Curran Associates.

\end{CSLReferences}

\section{Manuscripts in progress}\label{manuscripts-in-progress}

\phantomsection\label{refs-032e8b9e565c1aede97d12057428470a}
\begin{CSLReferences}{1}{0}
\bibitem[\citeproctext]{ref-11}
Luthra, S., Mechtenburg, H., Olson, H. E., Zaharchuk, H. A., \& Myers,
E. B. (2025). Stimulation of right temporal cortex enhances talker
typicality judgments {[}Manuscript submitted, under review{]}.
\emph{Journal of Cognitive Neuroscience}.

\bibitem[\citeproctext]{ref-9}
Zaharchuk, H. A., Olson, H. E., Mechtenburg, H., Phillips, M. C., \&
Myers, E. B. (2025). Measuring age-related differences in phonetic
gradiency with the visual analogue scale and eye tracking {[}Manuscript
submitted, in revision{]}. \emph{Journal of the Acoustical Society of
America}.

\bibitem[\citeproctext]{ref-7}
Zaharchuk, H. A., \& Van Hell, J. G. (2025). The effects of exposure
variability and similarity on talker-independent adaptation to
Spanish-accented English {[}Manuscript submitted, in revision{]}.
\emph{Journal of Memory and Language}.

\bibitem[\citeproctext]{ref-5}
Litcofsky, K., Zaharchuk, H. A., \& Van Hell, J. G. (2024).
Cross-modality structural priming in dialogue is influenced by
interlocutor accent {[}Manuscript submitted, in revision{]}.
\emph{Journal of Neurolinguistics}.

\bibitem[\citeproctext]{ref-6}
Zaharchuk, H. A., Medina, V., Paterno, S., Viswanathan, N., \& Van Hell,
J. G. (n.d.). Sentential context impacts bilingual speech-in-speech
recognition {[}Manuscript in preparation{]}.

\bibitem[\citeproctext]{ref-10}
Zaharchuk, H. A., \& Van Hell, J. G. (n.d.). The neurocognitive
correlates of perceptual adaptation to Spanish-accented English
{[}Manuscript in preparation{]}. \emph{Neuropsychologia}.

\end{CSLReferences}

\section{Invited talks}\label{invited-talks}

\phantomsection\label{refs-06ac7968e1e72a7b73a6c28c3943c4dd}
\begin{CSLReferences}{1}{0}
\bibitem[\citeproctext]{ref-4}
Zaharchuk, H. A. (2025). Bilingualism: Who, what, when, where, why, and
how. Guest lecture, University of Connecticut, March 3.

\bibitem[\citeproctext]{ref-2}
Zaharchuk, H. A. (2023). Sociolinguistics in the lab and the field.
Guest lecture, University at Buffalo, March 13.

\bibitem[\citeproctext]{ref-1}
Zaharchuk, H. A., Shevlin, A., \& Van Hell, J. G. (2022a). Are our
brains more prescriptive than our mouths? Experience with dialectal
variation in syntax differentially impacts ERPs and behavior. Lecture
series, Processing Variation (ProVAR), Federal University of Rio de
Janeiro, February 14.

\bibitem[\citeproctext]{ref-3}
Zaharchuk, H. A., Shevlin, A., \& Van Hell, J. G. (2022b). Experiencia
con variación dialéctica en sintaxis afecta el procesamiento de
lenguaje. Panel session, III Congreso Paraguayo de Lingüı́stica Aplicada
(CONPLA), Asunción, Paraguay, November 15-17.

\end{CSLReferences}

\section{Conference presentations}\label{conference-presentations}

\phantomsection\label{refs-5ba1e8a41031537c080c37a0821e8403}
\begin{CSLReferences}{1}{0}
\bibitem[\citeproctext]{ref-13}
Luthra, S., Mechtenberg, H., Olson, H. E., Zaharchuk, H. A., Feder, A.,
\& Myers, E. B. (2025, September). Evaluating a causal role for right
temporal cortex in accessing talker-specific phonetic signatures
{[}Poster session{]}. 17th Annual Meeting of the Society for the
Neurobiology of Language (SNL), Washington, DC.

\bibitem[\citeproctext]{ref-12}
Zaharchuk, H. A., Washington, P. N., \& Myers, E. B. (2025, September).
Hemispheric differences in speech perception: The role of right temporal
regions in talker-specific learning {[}Poster session{]}. 17th Annual
Meeting of the Society for the Neurobiology of Language (SNL),
Washington, DC.

\bibitem[\citeproctext]{ref-11}
Zaharchuk, H. A., \& Van Hell, J. G. (2024, November). The
neurocognitive correlates of talker-specific adaptation to
Spanish-accented speech {[}Poster session{]}. 65th Annual Meeting of the
Psychonomic Society, New York City, New York.

\bibitem[\citeproctext]{ref-10A}
Walker, A., Hampton, C., Wood, W., Miller, A. R., Zaharchuk, H. A., \&
Van Hell, J. G. (2024, April). Bidialectal brains: Profiles of
event-related potentials in a cross-dialectal listening task in Southern
US English speakers {[}Paper presentation{]}. Joint meeting of Language
Variation across the South V and the Southeastern Conference on
Linguistics 91.

\bibitem[\citeproctext]{ref-10}
Walker, A., Zaharchuk, H. A., Miller, A. R., \& Van Hell, J. G. (2023,
December). Your brain on accents: Profiles of event related potentials
in cross-dialectal listening in the US English context {[}Poster
session{]}. 185th Meeting of the Acoustical Society of America, Sydney,
Australia.

\bibitem[\citeproctext]{ref-9}
Zaharchuk, H. A., \& Van Hell, J. G. (2023, October). Investigating the
roles of acoustic-phonetic and lexico-semantic processing in resolving
perceptual ambiguity in Spanish-accented English {[}Poster session{]}.
15th Annual Meeting of the Society for the Neurobiology of Language
(SNL), Marseille, France.

\bibitem[\citeproctext]{ref-8}
Zaharchuk, H. A., Lipski, J., \& Van Hell, J. G. (2023, August). Mi casa
es tu posá: Exploring the bilingual mental lexicon in speakers of
Spanish and Palenquero {[}Poster session{]}. Architectures and
Mechanisms for Language Processing Conference (AMLaP 29), Donostia-San
Sebastián, Spain.

\bibitem[\citeproctext]{ref-6}
Zaharchuk, H. A., Medina, V., Paterno, S., Viswanathan, N., \& Van Hell,
J. G. (2023, March). Sentential context impacts bilingual
speech-in-speech recognition {[}Poster session{]}. 36th Annual
Conference on Human Sentence Processing (HSP), Pittsburgh, Pennsylvania.

\bibitem[\citeproctext]{ref-7}
Zaharchuk, H. A., Walker, A., \& Van Hell, J. G. (2023, March). Tracking
the time-course of cross-dialect comprehension with ERPs {[}Oral and
poster sessions{]}. 30th Annual Meeting of the Cognitive Neuroscience
Society (CNS), San Francisco, California.

\bibitem[\citeproctext]{ref-4}
Walker, A., Zaharchuk, H. A., Fernandez, C. B., \& Van Hell, J. G.
(2021, August). The role of dialect expectation on lexical processing:
EEG evidence {[}Oral session{]}. 5th Variation and Language Processing
Conference (VALP5), Copenhagen, Denmark.

\bibitem[\citeproctext]{ref-3}
Zaharchuk, H. A., Shevlin, A., \& Van Hell, J. G. (2020, May). We
{``might could''} revisit syntactic processing: Studying dialectal
variation with event-related potentials {[}Poster session{]}. 27th
Annual Meeting of the Cognitive Neuroscience Society (CNS), Boston,
Massachusetts.

\bibitem[\citeproctext]{ref-2}
Zaharchuk, H. A., Shevlin, A., \& Van Hell, J. G. (2019, November).
Auditory comprehension of double versus single modal constructions in
Mainstream American English listeners {[}Poster session{]}. 60th Annual
Meeting of the Psychonomic Society, Montréal, Québec, Canada.

\bibitem[\citeproctext]{ref-1}
Zaharchuk, H. A. (2015, June). Does linguistic experience change mental
representations of pitch? {[}Poster session{]}. University of Chicago
Undergraduate Research Symposium, Chicago, Illinois.

\end{CSLReferences}

\section{Teaching experience}\label{teaching-experience}

\begin{cventries}
    \cventry{Instructor}{Psychology of language}{The Pennsylvania State University}{Jan 2024 - May 2024}{\begin{cvitems}
\item Seminar course for upper-level psychology and linguistics undergraduates on psycholinguistics, with a focus on language development, bilingualism and second language acquisition, and linguistic diversity
\item Presented bi-weekly lectures to highlight key concepts and clarify models and methods; mentored students on interpreting empirical research through reflection papers, journal article presentations and discussions, and individual meetings; fostered student curiosity and creativity through final project proposal and poster presentation
\end{cvitems}}
    \cventry{Teaching assistant}{Psychology of language}{The Pennsylvania State University}{Jan 2021 - May 2021}{\begin{cvitems}
\item Instructor: Dr. Katharine Donnelly Adams
\item Seminar course covering auditory speech perception, visual word recognition, language representation, sentence comprehension and production, and language acquisition across the lifespan
\item Presented guest lecture on linguistic variation and the relation between production and comprehension
\end{cvitems}}
    \cventry{Teaching assistant}{Basic research methods in psychology}{The Pennsylvania State University}{Aug 2020 - Dec 2020}{\begin{cvitems}
\item Instructor: Dr. Nicholas Pearson
\item Introduction to experimental design, analysis, and presentation
\item Led weekly laboratory section to reinforce lecture concepts
\end{cvitems}}
    \cventry{Teaching assistant}{What makes us human?}{The Pennsylvania State University}{Jan 2020 - May 2020}{\begin{cvitems}
\item Instructor: Dr. Daniel Weiss
\item Senior seminar in comparative psychology
\end{cvitems}}
    \cventry{Teaching assistant}{Neurological bases of human behavior}{The Pennsylvania State University}{Aug 2019 - Dec 2019}{\begin{cvitems}
\item Instructor: Dr. Stephen Wilson
\item Introduction to neuroscience theories and methods
\item Presented guest lecture on auditory system and neuroanatomy of language production and comprehension
\end{cvitems}}
\end{cventries}

\newpage

\section{Academic service}\label{academic-service}

\begin{itemize}
\tightlist
\item
  Ad hoc reviewing

  \begin{itemize}
  \tightlist
  \item
    \emph{Brain and Language}
  \item
    \emph{Cognitive, Affective, and Behavioral Neuroscience}
  \item
    \emph{Journal of Experimental Psychology: Learning, Memory, and
    Cognition}
  \item
    \emph{Journal of Speech, Language, and Hearing Research}
  \item
    \emph{Perspectives}
  \item
    \emph{Psychonomic Bulletin \& Review}
  \item
    \emph{Scientific Reports}
  \end{itemize}
\item
  Workshops and tutorials

  \begin{itemize}
  \tightlist
  \item
    Leader:
    \href{https://github.com/hollzzar/markdown-tutorial}{\ul{R Markdown for Graduate Students}}
    (2020) for 20 students,
    \href{https://github.com/hollzzar/reports-with-markdown}{\ul{Creating HTML Reports with R Markdown}}
    (2020) for 10 students
  \item
    Assistant:
    \href{https://github.com/psu-psychology/r-bootcamp-2019}{\ul{R Bootcamp}}
    led by Dr.~Rick Gilmore for 50+ students
  \end{itemize}
\item
  \href{https://www.hzaharchuk.com/rmarkdown-guide/}{\ul{R Markdown for Psychology Graduate Students}}
  resource
\end{itemize}

\section{Outreach}\label{outreach}

\begin{itemize}
\tightlist
\item
  Developing a Community Engagement Training Program for researchers in
  collaboration with the
  \href{https://cncct.research.uconn.edu/}{\ul{NIH T32 Cognitive Neuroscience of Communication - Connecticut}}
  (CNC-CT) program
\item
  Volunteered with the
  \href{https://skoelab.speech-language-hearing.uconn.edu/}{\ul{Auditory Brain Research Lab}}
  to provide free hearing screenings in English and Spanish at the 4th
  Multicultural Latin Festival (July 2025, Willimantic, CT)
\item
  Mentored underrepresented and first-generation students applying for
  graduate programs in cognitive psychology as part of the
  \href{https://magic.initiative.uconn.edu/}{\ul{Mentoring Aspiring Graduate students and building an Inclusive Community}}
  (MAGIC) program at UConn (2024-2025)
\end{itemize}

\section{Skills}\label{skills}

\begin{itemize}
\tightlist
\item
  Natural languages: English, Spanish (B2/Intermediate)
\item
  Programming languages: R, MATLAB, Python, SQL
\item
  Experiment programming: E-Prime, PsychoPy, Gorilla, LabVanced,
  Pavlovia
\item
  Acoustic analysis: Praat
\end{itemize}

\section{Affiliations}\label{affiliations}

\begin{itemize}
\tightlist
\item
  Center for Language Science (Penn State)
\item
  Cognitive Neuroscience Society
\item
  Institute for the Brain and Cognitive Sciences (UConn)
\item
  Psychonomic Society
\item
  Society for the Neurobiology of Language
\item
  Women in Cognitive Science
\end{itemize}


\label{LastPage}~
\end{document}
